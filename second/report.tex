\documentclass[a4paper,12pt]{article}
\usepackage[utf8x]{inputenc}
\usepackage[T2A]{fontenc}
\usepackage[russian,english]{babel}
\usepackage{amsmath}
\usepackage{cmap}
\usepackage{booktabs}
\usepackage{caption}
\usepackage{enumitem}
\usepackage{caption}
\usepackage{listings}
\usepackage{xcolor}
\usepackage{setspace}
\usepackage[left=3cm, right=1.5cm, top=2cm, bottom=2cm]{geometry}
\renewcommand{\labelenumii}{\arabic{enumi}.\arabic{enumii}.}
\lstset{
    language=C++,
    basicstyle=\small\ttfamily,
    keywordstyle=\color{blue},
    commentstyle=\color{green!40!black},
    stringstyle=\color{purple},
    numbers=left,
    numberstyle=\tiny,
    numbersep=5pt,
    breaklines=true,
    frame=single,
    backgroundcolor=\color{gray!10},
    rulecolor=\color{black!30},
    showstringspaces=false,
    extendedchars=\true, % Включение расширенных символов, включая русский текст
}
\begin{document}

\begin{center}
\hfill \break
\textbf{\large{Министерство науки и высшего образования Российской Федерации\\
Федеральное государственное автономное образовательное\\
учреждение высшего образования}}
\\
\large{\textbf{«КАЗАНСКИЙ (ПРИВОЛЖСКИЙ) ФЕДЕРАЛЬНЫЙ УНИВЕРСИТЕТ»}}\\
\hfill \break
\large{ИНСТИТУТ ВЫЧИСЛИТЕЛЬНОЙ МАТЕМАТИКИ\\ И ИНФОРМАЦИОННЫХ ТЕХНОЛОГИЙ}\\
 \hfill \break
\large{Кафедра прикладной математики и искусственого интеллекта}\\
\hfill\break
\hfill \break
\large{Направление подготовки: 01.03.04 – Прикладная математика}\\
\hfill \break
\hfill \break
\textbf{\large{ОТЧЁТ}}\\
\large{По дисциплине <<Численные методы>>}\\
\large{на тему:}\\
\large{<<Вычисление интеграла с помощью квадратурных формул>>}\\
\hfill \break
\hfill \break
\end{center}

\hfill \break
\begin{flushright}
			
    \large{Выполнил:}
    
    \large{студент группы 09-222}
    
    \large{Фаррахова Л. Ф.}
    
    \large{Проверил:}
    
    \large{ассистент Глазырина О.В.}
    
\end{flushright}
\vfill
\begin{center} \large{Казань, 2024 год} \end{center}
\thispagestyle{empty}
 

\newpage
\begin{center}
\renewcommand{\contentsname}{Содержание}
\fontsize{14}{1.15}\selectfont
\mdseries\selectfont{\tableofcontents}
\newpage
\end{center}

\setlength{\parindent}{1.25cm}
\newpage
\selectfont\onehalfspacing{
\section{Постановка задачи}
\hspace{1.25cm}Необходимо изучить и сравнить различные способы приближённого вычисления функции
\begin{equation}
    Si(x) = \sum_{n=0}^{\infty} \frac{(-1)^n x^{2n+1}}{(2n+1)(2n+1)!}
\end{equation}
\begin{enumerate}
    \item Протабулировать Si(x) на отрезке [a,b] с шагом h и точностью $ \varepsilon$, основываясь на ряде Тейллора,
     предварительно вычислив его. Получив таким образом таблицу из 11 точек вида:
    %  \begin{equation}
    %     \text{erf}(x) =\displaystyle\frac{2}{\sqrt{\pi}} \displaystyle\sum\limits_{n=1}^{\infty}(-1)^n\displaystyle\frac{x^{2n+1}}{n!(2n+1)}
    %  \end{equation} Получив таким образом таблицу из 11 точек
    \begin{tabbing}
        $x_0$ \= $x_1$ \= $x_2$ \= \dots\\
        $f_0$ \> $f_1$ \> $f_2$ \> \dots\\
    \end{tabbing}
    $f_i = \text{Si}(x_i), \quad x_i = a + i  h, \quad i = 0,\dots,n.$
    \item {Вычислить Si(x) при помощи пяти составных квадратурных формул при\\
    \begin{equation*}
        h=(x_{i+1} - x_i)=\frac{b-a}{n}=\frac{\sin(t)}{t}
    \end{equation*}
        \begin{enumerate}
            \item {
                Формула правых прямоугольников:
                \begin{equation}
                    J_N(x) = \displaystyle\sum_{i=1}^{n}h g(x_i)
                \end{equation}
                }
            \item {
                Формула центральных прямоугольников:
                \begin{equation}
                    J_N(x) = \displaystyle\sum_{i=1}^{n}h g\left(\displaystyle\frac{x_i + x_{i+1}}{2}\right) 
                \end{equation}
                }
            \item {
                Формула трапеции:
                \begin{equation}
                    J_N(x) = \displaystyle\sum_{i=1}^{n}h \displaystyle\frac{g(x_i)+g(x_{i+1})}{2}
                \end{equation}
                }
            \item {
                Формула Симпсона:
                \begin{equation}
                    J_N(x) = \displaystyle\sum_{i=1}^{n}\displaystyle\frac{h}{6} \left[g(x_i)+4g\left(
                        \displaystyle\frac{x_i + x_{i+1}}{2}
                    \right) + g(x_{i+1}) \right]
                \end{equation}
                }
            \item {
                Формула Гаусса:
                \begin{equation}
                    J_N(x) = \displaystyle\sum_{i=1}^{n}\displaystyle\frac{h}{2} \left[g\left(
                        x_i + \displaystyle\frac{h}{2}\left(1 - \displaystyle\frac{1}{\sqrt{3}} \right)
                    \right) +
                    g\left(
                        x_i + \displaystyle\frac{h}{2}\left(1 + \displaystyle\frac{1}{\sqrt{3}} \right)
                    \right)
                    \right]
                \end{equation}
                }
        \end{enumerate}
    }
\end{enumerate}

Вычисления проводятся от начала интегрирования до каждой из 11 точек, увели\-чивая количество разбиений между точками в 2 раза до тех пор, пока погрешность больше $\varepsilon$.
\newpage
\section{Ход работы}
\hspace*{1.25cm}Для того чтобы найти значение функции в точке, необходимо протабулировать искомый интеграл на отрезке [a, b] с шагом $h = 0.4$ и точностью $\varepsilon$. 

Для этого выделим два общих члена $a_n, \; a_{n + 1}$ из $Si(x)$ и найдём\\ $q_n = \dfrac{a_{n+1}}{a_n}$:
    \begin{equation}
        a_{n}=\frac{(-1)^{n}x^{2n+1}}{(2n+1)(2n+1)!}, \quad a_{n+1}=\frac{(-1)^{n+1}x^{2n+3}}{(2n+3)(2n+3)!}.
    \end{equation}
    \begin{equation}
        q_{n}=\frac{-x^{2}(2n+1)}{(2n+2)(2n+3)^2}.
    \end{equation}
Вычислим по ней значения функции в 11 узлах интерполяции (Таблица 1).
\begin{table}[h]
    \centering
    \begin{tabular}{|c|c|}
         \hline
        $x_i$ & $erf(x_i)$\\
        \hline
        0,0 & 0,0000000000\\
        \hline
        0,4 & 0,3964614570\\
        \hline
        0,8 & 0,7720957994\\
        \hline
        1,2 & 1,1080472469\\
        \hline
        1,6 & 1,3891806602\\
        \hline
        2,0 & 1,6054129601\\
        \hline
        2,4 & 1,7524855137\\
        \hline
        2,8 & 1,8320965767\\
        \hline
        3,2 & 1,8514009714\\
        \hline
        3,6 & 1,8219480515\\
        \hline
        4,0 & 1,7582031488\\
        \hline
    \end{tabular}
    \caption*{\small{Таблица 1 - точки $x_i$ и значения разложения функции $Si(x_i)$}}
\end{table}
\clearpage
Далее вычислим $Si(x)$ с помощью пяти составных квадратурных формул и соста\-вим для каждой формулы таблицу, в которой первый столбец - одиннадцать точек разбиения, второй - значение интеграла, третий - значение интеграла, вычисленного соответсвующим методом. В четвертом столбце находятся значения погрешности. В последнем - количество разбиений, необходимых для подсчета интеграла с заданной точностью.
\begin{enumerate}[label = \arabic*.]
    \item {Правые прямоугольники:
        \begin{table}[h]
          \centering
          \begin{tabular}{|c|c|c|c|c|}
            \hline
            $x_i$ & $J_0(x_i)$ & $J_(x_i)$ & $\left|J_0(x_i) - J_N(x_i)\right|$ & $N$\\
            \hline
            0,0 & 0,0000000000 & 0,0000000000 & 0,0000000000 & 2 \\
            \hline
            0,4 & 0,3964614570 & 0,3964561820 & 0,0000052750 & 1024 \\
            \hline
            0,8 & 0,7720957994 & 0,7720555663 & 0,0000402331 & 1024 \\
            \hline
            1,2 & 1,1080472469 & 1,1079159975 & 0,0001312494 & 1024 \\
            \hline
            1,6 & 1,3891806602 & 1,3888883591 & 0,0002923012 & 1024 \\
            \hline
            2,0 & 1,6054129601 & 1,6048804522 & 0,0005325079 & 1024 \\
            \hline
            2,4 & 1,7524855137 & 1,7516450882 & 0,0008404255 & 1024 \\
            \hline
            2,8 & 1,8320965767 & 1,8308923244 & 0,0012042522 & 1024 \\
            \hline
            3,2 & 1,8514009714 & 1,8498086929 & 0,0015922785 & 1024 \\
            \hline
            3,6 & 1,8219480515 & 1,8199745417 & 0,0019735098 & 1024 \\
            \hline
            4,0 & 1,7582031488 & 1,7558794022 & 0,0023237467 & 1024 \\
            \hline

          \end{tabular}
          \caption*{\small{Таблица 2 - таблица значений для формулы правых прямоугольников}}
        \end{table}
    }
    \item {Центральные прямоугольники:
        \begin{table}[h]
          \centering
          \begin{tabular}{|c|c|c|c|c|}
            \hline
            $x_i$ & $J_0(x_i)$ & $J_(x_i)$ & $\left|J_0(x_i) - J_N(x_i)\right|$ & $N$\\
            \hline
            0,0 & 0,0000000000 & 0,0000000000 & 0,0000000000 & 2 \\
            \hline
            0,4 & 0,3964614570 & 0,3964616656 & 0,0000002086 & 64 \\
            \hline
            0,8 & 0,7720957994 & 0,7720960975 & 0,0000002980 & 256 \\
            \hline
            1,2 & 1,1080472469 & 1,1080478430 & 0,0000005960 & 512 \\
            \hline
            1,6 & 1,3891806602 & 1,3891806602 & 0,0000000000 & 1024 \\
            \hline
            2,0 & 1,6054129601 & 1,6054127216 & 0,0000002384 & 1024 \\
            \hline
            2,4 & 1,7524855137 & 1,7524878979 & 0,0000023842 & 1024 \\
            \hline
            2,8 & 1,8320965767 & 1,8320971727 & 0,0000005960 & 1024 \\
            \hline
            3,2 & 1,8514009714 & 1,8514013290 & 0,0000003576 & 512 \\
            \hline
            3,6 & 1,8219480515 & 1,8219479322 & 0,0000001192 & 1024 \\
            \hline
            4,0 & 1,7582031488 & 1,7582037449 & 0,0000005960 & 512 \\
            \hline

          \end{tabular}
          \caption*{\small{Таблица 3 - таблица значений для формулы центральных прямоугольников}}
        \end{table}
        \clearpage
    }
    \item {Формула трапеций:
        \begin{table}[h]
          \centering
          \begin{tabular}{|c|c|c|c|c|}
            \hline
            $x_i$ & $J_0(x_i)$ & $J_(x_i)$ & $\left|J_0(x_i) - J_N(x_i)\right|$ & $N$\\
            \hline
            0,0 & 0,0000000000 & 0,0000000000 & 0,0000000000 & 2 \\
            \hline
            0,4 & 0,3964614570 & 0,3964614570 & 0,0000000000 & 128 \\
            \hline
            0,8 & 0,7720957994 & 0,7720955014 & 0,0000002980 & 256 \\
            \hline
            1,2 & 1,1080472469 & 1,1080472469 & 0,0000000000 & 512 \\
            \hline
            1,6 & 1,3891806602 & 1,3891803026 & 0,0000003576 & 1024 \\
            \hline
            2,0 & 1,6054129601 & 1,6054120064 & 0,0000009537 & 512 \\
            \hline
            2,4 & 1,7524855137 & 1,7524851561 & 0,0000003576 & 1024 \\
            \hline
            2,8 & 1,8320965767 & 1,8320960999 & 0,0000004768 & 1024 \\
            \hline
            3,2 & 1,8514009714 & 1,8514009714 & 0,0000000000 & 1024 \\
            \hline
            3,6 & 1,8219480515 & 1,8219457865 & 0,0000022650 & 1024 \\
            \hline
            4,0 & 1,7582031488 & 1,7582020760 & 0,0000010729 & 1024 \\
            \hline
          \end{tabular}
          \caption*{\small{Таблица 4 - таблица значений для формулы трапеций}}
        \end{table}
    }
    \item Формула Симпсона
        \begin{enumerate}
            \item {Вывод формулы Симпсона через интегральный полином Лагранжа:\\
            Формула для полинома Лагранжа:
            \begin{equation}
                L_n(x) = \sum_{i=0}^{n}f(x_i)\prod_{i \ne j, j = 0}^{n}\frac{x - x_j}{x_i - x_j}
            \end{equation}
            По трём узлам $(x_1 = a, x_2 = \dfrac{a+b}{2}, x_3 = b):
            L_2 = f(a)\left(\dfrac{x - \dfrac{a+b}{2}}{a - \dfrac{a+b}{2}}\right)\left(\dfrac{x - b}{a - b}\right)+\\
            f\left(\dfrac{a +b}{2}\right)\left(\dfrac{x-a}{\dfrac{a+b}{2} - a}\right)\left(\dfrac{x-b}{\dfrac{a+b}{2} - b}\right)+ f(b)\left(\dfrac{x - \dfrac{a+b}{2}}{b - \dfrac{a+b}{2}}\right)\left(\dfrac{x - b}{b - a}\right).$\\
            \hfill\break
            Проинтегрируем выражение по интервалу [a,b]:
            \begin{equation}
                \int\limits_{a}^{b}L_2(x)\mathrm{d}x = f(a)c_1 + f\left(\frac{a+b}{2}\right)c_2 + f(b)c_3
            \end{equation}
            где $c_1 = \dfrac{b-a}{6}, c_2 = \dfrac{2}{3}(b - a), c_3 = \dfrac{b-a}{6}.$\\
            \hfill\break
            Тогда:
            \begin{equation}
                \int\limits_{a}^{b}L_2(x)\mathrm{d}x = \frac{b-a}{6}\left(f(a) + 4f\left(\frac{a+b}{2}\right)+f(b)\right)
            \end{equation}
            \clearpage
            }
            \item {Значения, полученные для формулы Симпсона:
            \begin{table}[h]
            \centering
            \begin{tabular}{|c|c|c|c|c|}
                \hline
                $x_i$ & $J_0(x_i)$ & $J_(x_i)$ & $\left|J_0(x_i) - J_N(x_i)\right|$ & $N$\\
                \hline
                0,0 & 0,0000000000 & 0,0000000000 & 0,0000000000 & 2 \\
                \hline
                0,4 & 0,3964614570 & 0,3964614868 & 0,0000000298 & 4 \\
                \hline
                0,8 & 0,7720957994 & 0,7720957398 & 0,0000000596 & 8 \\
                \hline
                1,2 & 1,1080472469 & 1,1080472469 & 0,0000000000 & 8 \\
                \hline
                1,6 & 1,3891806602 & 1,3891803026 & 0,0000003576 & 16 \\
                \hline
                2,0 & 1,6054129601 & 1,6054130793 & 0,0000001192 & 16 \\
                \hline
                2,4 & 1,7524855137 & 1,7524855137 & 0,0000000000 & 16 \\
                \hline
                2,8 & 1,8320965767 & 1,8320968151 & 0,0000002384 & 16 \\
                \hline
                3,2 & 1,8514009714 & 1,8514008522 & 0,0000001192 & 32 \\
                \hline
                3,6 & 1,8219480515 & 1,8219482899 & 0,0000002384 & 16 \\
                \hline
                4,0 & 1,7582031488 & 1,7582032681 & 0,0000001192 & 16 \\
                \hline

            \end{tabular}
            \caption*{\small{Таблица 5 - таблица значений для формулы Симпсона}}
            \end{table}
            }
        \end{enumerate}
    \item {Формула Гаусса:
        \begin{table}[h]
          \centering
          \begin{tabular}{|c|c|c|c|c|}
            \hline
            $x_i$ & $J_0(x_i)$ & $J_(x_i)$ & $\left|J_0(x_i) - J_N(x_i)\right|$ & $N$\\
            \hline
            0,0 & 0,0000000000 & 0,0000000000 & 0,0000000000 & 2 \\
            \hline
            0,4 & 0,3964614570 & 0,3964614868 & 0,0000000298 & 4 \\
            \hline
            0,8 & 0,7720957994 & 0,7720956802 & 0,0000001192 & 4 \\
            \hline
            1,2 & 1,1080472469 & 1,1080472469 & 0,0000000000 & 8 \\
            \hline
            1,6 & 1,3891806602 & 1,3891805410 & 0,0000001192 & 16 \\
            \hline
            2,0 & 1,6054129601 & 1,6054128408 & 0,0000001192 & 16 \\
            \hline
            2,4 & 1,7524855137 & 1,7524856329 & 0,0000001192 & 16 \\
            \hline
            2,8 & 1,8320965767 & 1,8320965767 & 0,0000000000 & 16 \\
            \hline
            3,2 & 1,8514009714 & 1,8514007330 & 0,0000002384 & 16 \\
            \hline
            3,6 & 1,8219480515 & 1,8219481707 & 0,0000001192 & 16 \\
            \hline
            4,0 & 1,7582031488 & 1,7582032681 & 0,0000001192 & 16 \\
            \hline

          \end{tabular}
          \caption*{\small{Таблица 6 - таблица значений для формулы Гаусса}}
        \end{table}
    }
\end{enumerate}
\section{Выводы}
\hspace{1.25cm}В ходе работы были изучены численные методы вычисления интегралов с приме\-нением пяти различных квадратурных формул. Анализ результатов показал, что методы Гаусса и Симпсона являются наиболее эффективными. Они обеспечивает высокую точность при минимальном числе разбиений благодаря специально подобранным узлам для приближенного вычисления интеграла.
\clearpage
\section{Листинг программы}
\lstinputlisting[language=C++]{./main.cpp}
\end{document}