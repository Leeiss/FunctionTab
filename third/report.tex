\documentclass[a4paper,12pt]{article}
\usepackage[utf8x]{inputenc}
\usepackage[T2A]{fontenc}
\usepackage[russian,english]{babel}
\usepackage{amsmath}
\usepackage{cmap}
\usepackage{booktabs}
\usepackage{caption}
\usepackage{enumitem}
\usepackage{listings}
\usepackage{xcolor}
\usepackage{setspace}
\usepackage[left=2cm, right=1.5cm, top=2cm, bottom=2cm]{geometry}
\renewcommand{\labelenumii}{\arabic{enumi}.\arabic{enumii}.}
\lstset{
    language=C++,
    basicstyle=\small\ttfamily,
    keywordstyle=\color{blue},
    commentstyle=\color{green!40!black},
    stringstyle=\color{purple},
    numbers=left,
    numberstyle=\tiny,
    numbersep=5pt,
    breaklines=true,
    frame=single,
    backgroundcolor=\color{gray!10},
    rulecolor=\color{black!30},
    showstringspaces=false,
    extendedchars=\true, % Включение расширенных символов, включая русский текст
}
\begin{document}

\begin{center}
\hfill \break
\textbf{\large{Министерство науки и высшего образования Российской Федерации\\
Федеральное государственное автономное образовательное\\
учреждение высшего образования}}
\\
\large{\textbf{«КАЗАНСКИЙ (ПРИВОЛЖСКИЙ) ФЕДЕРАЛЬНЫЙ УНИВЕРСИТЕТ»}}\\
\hfill \break
\large{ИНСТИТУТ ВЫЧИСЛИТЕЛЬНОЙ МАТЕМАТИКИ\\ И ИНФОРМАЦИОННЫХ ТЕХНОЛОГИЙ}\\
\hfill \break
\large{Кафедра прикладной математики и искусственого интеллекта}\\
\hfill\break
\hfill \break
\large{Направление подготовки: 01.03.04 – Прикладная математика}\\
\hfill \break
\hfill \break
\textbf{\large{ОТЧЁТ}}\\
\large{По дисциплине <<Численные методы>>}\\
\large{на тему:}\\
\large{<<Система линейных алгебраических уравнений>>}\\
\hfill \break
\hfill \break
\end{center}

\hfill \break
\begin{flushright}
			
    \large{Выполнил:}
    
    \large{студент группы 09-222}
    
    \large{Фаррахова. Л.Ф.}
    
    \large{Проверил:}
    
    \large{ассистент Глазырина О.В.}
    
\end{flushright}
\vfill
\begin{center} \large{Казань, 2024 год} \end{center}
\thispagestyle{empty}
 

\newpage
\begin{center}
\renewcommand{\contentsname}{Содержание}
\fontsize{14}{1.15}\selectfont
\mdseries\selectfont{\tableofcontents}
\newpage
\end{center}
\setlength{\parindent}{1.25cm}
\newpage
\selectfont\onehalfspacing{
\section{Постановка задачи}
\hspace{1.25cm}Решить систему линейных алгебраических уравнений:
\begin{equation}\label{eq:main_sys}
    \begin{cases}
        &(a_1 + a_2 + h^2g_1)y_1 - a_2y_2 = f_1h^2,\\
        &\hspace{1cm}\dots \quad \dots \quad \dots \quad \dots\\
        &-a_iy_{i-1} + (a_i + a_{i+1} + h^2g_i)y_i - a_{i+1}y_{i+1} = f_ih^2,\\
        &\hspace{1cm}\dots \quad \dots \quad \dots \quad \dots\\
        &(a_{n-1} + a_{n} + h^2g_{n-1})y_{n-1} - a_{n-1}y_{n-2} = f_{n-1}h^2.\\
    \end{cases}
\end{equation}

Здесь $a_i = p(ih),\;g_i = q(ih),\;f_i = f(ih),\;f(x) = -(p(x)u'(x))' + q(x)u(x),\;\\h = 1/n,\;p,\;q,\;u\;~-$ заданные функции.

Для этого использовать метод прогонки и итерационные методы:
\begin{enumerate}[label = \arabic*.]
    \item метод Зейделя.
    \item метод нижней релаксации.
    \item метод наискорейшего спуска.
\end{enumerate}

Продолжать вычисления в итерационных методах, пока не выполнится условие:
\begin{equation*}
    \max_{1 \le i \le n - 1} \left|r_{i}^{k}\right| \le \varepsilon,
\end{equation*}
$r\;~-$ вектор невязки, $\varepsilon\;~-$ заданное число.

\textbf{Исходные данные:} $n = 10,\; n = 50,\;\varepsilon = h^3,\;u(x) = x^{\alpha}(1-x)^{\beta},\;\\p(x) = 1 + x^{\gamma},\;g(x) = x + 1,\;\alpha = 4,\;\beta = 1,\;\gamma = 1.$

Сравнить результаты вычислений, составив соответсвтенные таблицы и найти наилуч\-ший из методов решения.
\section{Ход работы}
\subsection{Метод прогонки}
\hspace*{1.25cm} Метод прогонки состоит из двух этапов: прямой ход (определение прогоночных коэффициентов), 
обратных ход (вычисление неизвестных $y_i$).

Основным преимуществом является экономичность, ведь метод максимально использует структуру исходной системы.

К недостаткам же можно отнести то, что с каждой итерацией накапливается ошибка округления.

Прямой ход метода заключается в нахождении прогоночных коэффициентов: 
\begin{equation}
    \begin{cases}
        \alpha_1 = \dfrac{a_1}{a_0 + a_1 + h^2g_1},\\
        \alpha_{i+1} = \dfrac{a_{i+1}}{a_i + a_{i+1} + h^2g_i - \alpha_ia_i}, \quad i = \overline{2, n - 1};\\
    \end{cases}
\end{equation}
\begin{equation}
    \begin{cases}
        \beta_1 = \dfrac{f_0h^2}{a_0 + a_1 + h^2g_1},\\
        \beta_{i+1} = \dfrac{f_ih^2 + \beta_ia_i}{a_i + a_{i+1} + h^2g_i - \alpha_ia_i}, \quad i = \overline{2, n - 1};\\
    \end{cases}
\end{equation}

Обратный ход метода - вычисление формул для нахождения неизвестных:
\begin{equation}
    \begin{cases}
        y_n = \beta_{n+1};\\
        y_i = \alpha_{i+1}y_{i+1} + \beta_{i+1}, \quad i = \overline{n-1, 0};\\
    \end{cases}
\end{equation}

Формулы (2-4) являются методом Гаусса, записанным применительно трёх\-диаго\-наль\-ной системы уравнений. 
Метод может быть реализован только в случае, когда в формулах (3) и (4) все знаменатели отличны от нуля, 
то есть условие выполняется, когда матрица системы (2) имеет диагональное препобладание.

Проделаем вычисления и составим таблицу для $n = 10$ (Таблица 1), для $n = 50$ (Таблица 2):
\begin{table}[h]
    \centering
    \begin{tabular}{|c|c|c|c|}
        \hline
        $ih$ & $y_i$ & $u(ih)$ & $\left|y_i - u(ih)\right|$\\
        \hline
        0,0 & 0,000000  & 0,000000  & 0,000000  \\           
        \hline
        0,1 & 0,014011  & 0,000090  &  0,013921 \\
        \hline
        0,2 & 0,027929 &  0,001280 &  0,026649 \\
        \hline
        0,3 &  0,044160 & 0,005670  &  0,038490 \\
        \hline
        0,4 & 0,065086  & 0,015360  &  0,049726 \\
        \hline
        0,5 & 0,091907  &  0,031250 &  0,060657 \\
        \hline
        0,6 &  0,123480 &  0,051840 &  0,071640 \\
        \hline
        0,7 &  0,155141 & 0,072030  &  0,083111 \\
        \hline
        0,8 & 0,177525  & 0,081920  &  0,095605 \\
        \hline
        0,9 & 0,175383  & 0,065610  &  0,109773 \\
        \hline
        1,0 & 0,126400  & 0,000000  &  0,126400 \\
        \hline
    \end{tabular}
    \caption*{\small{Таблица 1 - значения метода прогонки для $n = 10$}}
\end{table}
\clearpage
\begin{table}[h]
    \centering
    \begin{tabular}{|c|c|c|c|}
        \hline
        $ih$ & $y_i$ & $u(ih)$ & $\left|y_i - u(ih)\right|$\\
        \hline
		0.1 & 0.002538 & 0.000090 & 0.002448 \\ \hline
		0.2 & 0.005974 & 0.001280 & 0.004694 \\ \hline
		0.3 & 0.012450 & 0.005670 & 0.006780 \\ \hline
		0.4 & 0.024101 & 0.015360 & 0.008741 \\ \hline
		0.5 & 0.041867 & 0.031250 & 0.010617 \\ \hline
		0.6 & 0.064297 & 0.051840 & 0.012457 \\ \hline
		0.7 & 0.086357 & 0.072030 & 0.014327 \\ \hline
		0.8 & 0.098232 & 0.081920 & 0.016312 \\ \hline
		0.9 & 0.084129 & 0.065610 & 0.018519 \\ \hline
		1.0 & 0.021080 & 0.000000 & 0.021080 \\ \hline

    \end{tabular}
    \caption*{\small{Таблица 2 - значения метода прогонки для $n = 50$}}
\end{table}

\subsection{Метод Якоби}
\hspace*{1.25cm}Для больших систем вида $Ax=b$ предпочтительнее оказываются итерационные методы. 
Основная идея данных методов состоит в построении последовательности векторов $x^k, \; k=1,2,\dots,$ 
сходящейся к решению исходной системы. За приближенное решение принимается вектор $x^k$ при достаточно большом $k$.

Будем считать, что все диагональные элементы матрицы $А$ из полной системы $Ax=b$ отличны от нуля. 
Представим эту систему, разрешая каждое уравнение относительно переменной, стоящей на главной диагонали: 
\begin{equation}
x_i = - \sum\limits_{j=1}^{i-1} \frac{a_{ij}}{a_{ii}}x_j - \sum\limits_{j=i+1}^n \frac{a_{ij}}{a_{ii}}x_j+\frac{b_i}{a_{ii}}, \quad i = \overline{1,n}.
 \label{5}
\end{equation}

Выберем некоторое начальное приближение $x^0=(x_1^0, x_2^0, {\dots}, x_n^0)^T$. 
Построим последова\-тель\-ность векторов $x^1, x^2, {\dots},$ определяя вектор $x^{k+1}$ по уже найденному вектору $x^k$ при помощи соотношения: 
\begin{equation}
x_i^{k+1}= - \sum\limits_{j=1}^{i-1} \frac{a_{ij}}{a_{ii}}x_j^k - \sum\limits_{j=i+1}^n \frac{a_{ij}}{a_{ii}}x_j^k+\frac{b_i}{a_{ii}}, \quad i = \overline{1,n}. 
 \label{6}
\end{equation}

Формула (\ref{6}) определяют итерационный метод решения системы (\ref{5}), называемый мето\-дом Якоби или методом простой итерации.

Запишем этот метод для нашей системы:
\begin{equation}
    y_i^{k+1}=  -\sum\limits_{j=1}^{i-1}\frac{a_j}{a_i+a_{i+1}+h^2 g_i}y_j^{k} \; - \;\sum\limits_{j=i+1}^{n}\frac{a_{j}}{a_i+a_{i+1}+h^2 g_i}y_j^k\; +\; \frac{f_i h^2}{a_i+a_{i+1}+h^2 g_i},$$$$
    i=\overline{1, n-1};
     \label{8}
\end{equation}

Вычисления продолжаем, пока не выполнится условие:
$$\max\limits |r_i^k| \leq \varepsilon,$$
где $r^k ~-$ вектор невязки для $k$-той итерации $r^k=Ay^k-f, \quad \varepsilon=h^3.$

Составим таблицы вычисленных результатов для $n = 10$, для $n = 50$, 
в которых будем сравнивать значения метода прогонки и метода Якоби для точки $i,\; i = \overline{0, n - 1}$, 
найдём модуль их разности и значение $k$, при котором была достигнута необходимая точность.
\begin{table}[h]
    \centering
    \begin{tabular}{|c|c|c|c|c|}
        \hline
        $i$ & $y_i$ & $y_i^k$ & $\left|y_i - y_i^k\right|$ & $k$\\
        \hline
		0 &  0,014011 &  0,011380 &  0,002631 & 36 \\ \hline
		1 &  0,027929 &  0,023304 &  0,004625 & 36 \\ \hline
		2 &  0,044160 &  0,038220 &  0,005940 & 36 \\ \hline
		3 &  0,065086 &  0,058498 &  0,006587 & 36 \\ \hline
		4 &  0,091907 &  0,085288 &  0,006619 & 36 \\ \hline
		5 &  0,123480 &  0,117355 &  0,006125 & 36 \\ \hline
		6 &  0,155141 &  0,149921 &  0,005219 & 36 \\ \hline
		7 &  0,177525 &  0,173497 &  0,004027 & 36 \\ \hline
		8 &  0,175383 &  0,172703 &  0,002680 & 36 \\ \hline
		9 &  0,126400 &  0,125099 &  0,001301 & 36 \\ \hline
    \end{tabular}
    \caption*{\small{Таблица 3 - значения метода Якоби для $n = 10$}}
\end{table}
\begin{table}[h]
    \centering
    \begin{tabular}{|c|c|c|c|c|}
        \hline
        $i$ & $y_i$ & $y_i^k$ & $\left|y_i - y_i^k\right|$ & $k$\\
        \hline
	    0 & 0,000509 & 0,000408 & 0,000100 & 955 \\ \hline
	    4 & 0,002538 & 0,002067 & 0,000471 & 955 \\ \hline
	    8 & 0,005120 & 0,004337 & 0,000783 & 955 \\ \hline
	   12 & 0,009344 & 0,008320 & 0,001024 & 955 \\ \hline
	   16 & 0,016397 & 0,015209 & 0,001188 & 955 \\ \hline
	   20 & 0,027169 & 0,025898 & 0,001271 & 955 \\ \hline
	   24 & 0,041867 & 0,040593 & 0,001274 & 955 \\ \hline
	   28 & 0,059618 & 0,058415 & 0,001203 & 955 \\ \hline
	   32 & 0,078085 & 0,077019 & 0,001066 & 955 \\ \hline
	   36 & 0,093067 & 0,092191 & 0,000876 & 955 \\ \hline
	   40 & 0,098113 & 0,097467 & 0,000646 & 955 \\ \hline
	   45 & 0,076309 & 0,075983 & 0,000327 & 955 \\ \hline
	   49 & 0,021080 & 0,021016 & 0,000064 & 955 \\ \hline
	   
    \end{tabular}
    \caption*{\small{Таблица 4 - значения метода Якоби для $n = 50$}}
\end{table}
\clearpage
\subsection{Метод нижней релаксации}
\hspace{1.25cm}Во многих ситуациях существенного ускорения сходимости можно добиться за счет введения так называемого итерационного параметра. 
Рассмотрим итерационный процесс:
\begin{equation}
    x_i^{k+1}=(1-\omega)x_i^k+\omega \bigg( - \sum\limits_{j=1}^{i-1} \frac{a_{ij}}{a_{ii}}x_j^{k+1}- \sum\limits_{j=i+1}^{n} \frac{a_{ij}}{a_{ii}}x_j^{k}+\frac{b_i}{a_{ii}}\bigg),
    $$$$ i=1,2, \dots, n, \quad k=0,1, \dots \; .
    \label{9}
\end{equation}

Этот метод называется методом релаксации -- одним из наиболее эффективных и широко используемых итерационных 
методов для решения систем линейных алгебраичес\-ких уравнений. Значение $\omega $ -- называется релаксационным параметром. 
При $\omega = 1$ метод переходит в метод Зейделя.  При $\omega \in (0,1)$ -- метод нижней релаксации, при $\omega \in (1,2)$ -- это метод верхней релаксации. 

Преобразуем формулу (9) относительно нашей системы:
\begin{equation}
    y_i^{k+1}=(1-\omega)y_i^k+\omega  \bigg(-\sum\limits_{j=1}^{i-1}\frac{a_j}{a_i+a_{i+1}+h^2 g_i}y_j^{k} \; - \;\sum\limits_{j=i+1}^{n}\frac{a_{j}}{a_i+a_{i+1}+h^2 g_i}y_j^k\; +\; \frac{f_i h^2}{a_i+a_{i+1}+h^2 g_i}\bigg),$$$$
    i=\overline{1, n-1};
    \label{10}
\end{equation}

Параметр $\omega$ следует выбирать так, чтобы метод релаксации сходился наиболее быстро. 
Нужно отметить, что оптимальный параметр для метода верхней релаксации лежит вблизи 0,8. 
Заполним таблицы, в которых приведём значения параметра $\omega$ и количство итераций $k$:
\begin{table}[h]
    \centering
    \begin{tabular}{|c|c|}
        \hline
        $\omega$ & $k$\\
        \hline
        0.1 & 63\\ \hline
        0.2 & 49\\ \hline
        0.3 & 56\\ \hline
        0.4 & 40\\ \hline
        0.5 & 32\\ \hline
        0.6 & 26\\ \hline
        0.7 & 24\\ \hline
        0.8 & 27\\ \hline
        0.9 & 34\\ \hline
    \end{tabular}
    \caption*{\small{Таблица 5 - значения $\omega$ и соответствующие значения $k$ для $n = 10$}}
\end{table}
\clearpage
\begin{table}[h]
    \centering
    \begin{tabular}{|c|c|}
        \hline
        $\omega$ & $k$\\
        \hline
        0.02 &  892\\ \hline
        0.14 &  721\\ \hline
        0.30 &  452\\ \hline
        0.38 &  228\\ \hline
        0.54 &  190\\ \hline
        0.62 &  141\\ \hline
        0.78 &  125\\ \hline
        0.84 &  117\\ \hline
        0.90 &  144\\ \hline
		0.94 &  159\\ \hline
    \end{tabular}
    \caption*{\small{Таблица 6 - значения $\omega$ и соответствующие значения $k$ для $n = 50$}}
\end{table}

Для вычислений выберем $\omega = 0,84$. Составим таблицы результатов для $n = 10$, $n = 50$,
в которых будем сравнивать значения метода прогонки и метода верхней релаксации для точки $i,\; i = \overline{0, n-1}$,
найдём модуль их разности и значение $k$:
\begin{table}[h]
    \centering
    \begin{tabular}{|c|c|c|c|c|}
        \hline
        $i$ & $y_i$ & $y_i^k$ & $\left|y_i - y_i^k\right|$ & $k$\\
        \hline
		0 & 0,014011 &  0,013805 &  0,000207 & 27 \\ \hline
		1 & 0,027929 &  0,027484 &  0,000444 & 27 \\ \hline
		2 & 0,044160 &  0,043500 &  0,000660 & 27 \\ \hline
		3 & 0,065086 &  0,064323 &  0,000762 & 27 \\ \hline
		4 & 0,091907 &  0,091294 &  0,000613 & 27 \\ \hline
		5 & 0,123480 &  0,123466 &  0,000014 & 27 \\ \hline
		6 & 0,155141 &  0,155153 &  0,000012 & 27 \\ \hline
		7 & 0,177525 &  0,177554 &  0,000029 & 27 \\ \hline
		8 & 0,175383 &  0,175418 &  0,000034 & 27 \\ \hline
		9 & 0,126400 &  0,126425 &  0,000025 & 27 \\ \hline
    \end{tabular}
    \caption*{\small{Таблица 7 - значения метода нижней релаксации для $n = 10$}}
\end{table}
\clearpage
\begin{table}[h]
    \centering
    \begin{tabular}{|c|c|c|c|c|}
        \hline
        $i$ & $y_i$ & $y_i^k$ & $\left|y_i - y_i^k\right|$ & $k$\\
        \hline
	    0 & 0,000509 & 0,000387 & 0,000122 & 117 \\ \hline
	    4 & 0,002538 & 0,002007 & 0,000531 & 117 \\ \hline
	    8 & 0,005120 & 0,004300 & 0,000820 & 117 \\ \hline
	   12 & 0,009344 & 0,008347 & 0,000997 & 117 \\ \hline
	   16 & 0,016397 & 0,015323 & 0,001073 & 117 \\ \hline
	   20 & 0,027169 & 0,026103 & 0,001066 & 117 \\ \hline
	   24 & 0,041867 & 0,040874 & 0,000993 & 117 \\ \hline
	   28 & 0,059618 & 0,058747 & 0,000871 & 117 \\ \hline
	   32 & 0,078085 & 0,077368 & 0,000717 & 117 \\ \hline
	   36 & 0,093067 & 0,092520 & 0,000547 & 117 \\ \hline
	   40 & 0,098113 & 0,097739 & 0,000375 & 117 \\ \hline
	   45 & 0,076309 & 0,076137 & 0,000173 & 117 \\ \hline
	   49 & 0,021080 & 0,021049 & 0,000032 & 117 \\ \hline
    \end{tabular}
    \caption*{\small{Таблица 8 - значения метода нижней релаксации для $n = 50$}}
\end{table}
\subsection{Метод наискорейшего спуска}
\hspace{1.25cm}Метод наискорейшего спуска для решений систем линейных алгебраических уравнений заключается 
в итерационном процессе, направленном на минимизацию квадратичной функции ошибки.

$x_k$ - вычислено, тогда

\begin{equation}
   x^{k+1} = x^k - \tau r_i^k, где i = \overline{1, n}, k = 0,1,...
    \label{11}
\end{equation}

\begin{align*}
    r^k &= Ax^k - b, \tau = \frac{(r^k, r^k)}{(Ar^k, r^k)}, (r^k, t^k) = \sum\limits_{i=1}^{n-1}(r_i^k)^2, (Ar^k)_i = -a_ir_{i-1}^k + (a_i + a_{i+1} + \\
    &+ h^2g_i)r_i^k - a_{i+1}r_{i+1}^k
\end{align*}
   
$r_i^k$ - вектор невязки для конкретного уравненения:
\begin{equation}
    r_i^k = -a_ix_{i-1}^k + (a_i + a_{i+1} + h^2g_i - a_{i+1}^k - f_ih^2)
     \label{12}
 \end{equation}

По сравнению с методом Якоби этот метод требует на каждом шаге 
итераций проведения дополнительной работы по вычислению параметра $\tau$. Вследствие
этого происходит адаптация к оптимальной скорости сходимости.

Составим таблицы результатов для $n = 10$, $n = 50$,
в которых будем сравнивать значения метода прогонки и метода верхней релаксации для точки $i,\; i = \overline{0, n-1}$,
найдём модуль их разности и значение $k$:
\begin{table}[h]
    \centering
    \begin{tabular}{|c|c|c|c|c|}
        \hline
        $i$ & $y_i$ & $y_i^k$ & $\left|y_i - y_i^k\right|$ & $k$\\
        \hline
		0 & 0,014011 & 0,012035 & 0,001977 & 38 \\ \hline
		1 & 0,027929 & 0,024447 & 0,003482 & 38 \\ \hline
		2 & 0,044160 & 0,039702 & 0,004458 & 38 \\ \hline
		3 & 0,065086 & 0,060300 & 0,004785 & 38 \\ \hline
		4 & 0,091907 & 0,087169 & 0,004738 & 38 \\ \hline
		5 & 0,123480 & 0,119300 & 0,004180 & 38 \\ \hline
		6 & 0,155141 & 0,151830 & 0,003310 & 38 \\ \hline
		7 & 0,177525 & 0,175069 & 0,002455 & 38 \\ \hline
		8 & 0,175383 & 0,174025 & 0,001358 & 38 \\ \hline
		9 & 0,126400 & 0,125790 & 0,000609 & 38 \\ \hline
    \end{tabular}
    \caption*{\small{Таблица 9 - значения метода наискорейшего спуска для $n = 10$}}
\end{table}
\begin{table}[h]
    \centering
    \begin{tabular}{|c|c|c|c|c|}
        \hline
        $i$ & $y_i$ & $y_i^k$ & $\left|y_i - y_i^k\right|$ & $k$\\
        \hline
	    0 & 0,000509 & 0,000487 & 0,000022 & 2549 \\ \hline
	    4 & 0,002538 & 0,002435 & 0,000103 & 2549 \\ \hline
	    8 & 0,005120 & 0,004951 & 0,000168 & 2549 \\ \hline
	   12 & 0,009344 & 0,009128 & 0,000215 & 2549 \\ \hline 
	   16 & 0,016397 & 0,016153 & 0,000243 & 2549 \\ \hline
	   20 & 0,027169 & 0,026916 & 0,000253 & 2549 \\ \hline
	   24 & 0,041867 & 0,041620 & 0,000247 & 2549 \\ \hline
	   32 & 0,078085 & 0,077889 & 0,000196 & 2549 \\ \hline
	   36 & 0,093067 & 0,092909 & 0,000157 & 2549 \\ \hline
	   40 & 0,098113 & 0,097999 & 0,000114 & 2549 \\ \hline
	   44 & 0,084129 & 0,084060 & 0,000069 & 2549 \\ \hline
	   49 & 0,021080 & 0,021067 & 0,000013 & 2549 \\ \hline

    \end{tabular}
    \caption*{\small{Таблица 10 - значения метода наискорейшего спуска для $n = 50$}}
\end{table}
\clearpage
\section{Выводы}
\hspace{1.25cm}В процессе выполнения работы были изучены методы решения заданной системы линейных алгебраических 
уравнений метод прогонки, итерационными методами: Якоби, нижней релаксации, наискорейшего спуска.

В результате наилучшим способом показал себя  метод верхней релаксации при итера\-ционном параметре $\omega = 1,84$. 
Данный метод показывается наилучшие результаты вычисле\-ния корней системы за наименьшее количество итераций.
}
\clearpage
\section{Листинг программы}
\lstinputlisting[language=C++]{./main.cpp}
\end{document}